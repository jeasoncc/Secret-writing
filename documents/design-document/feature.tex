% Created 2023-09-06 三 16:49
% Intended LaTeX compiler: pdflatex
\documentclass[11pt]{article}
\usepackage[utf8]{inputenc}
\usepackage[T1]{fontenc}
\usepackage{graphicx}
\usepackage{longtable}
\usepackage{wrapfig}
\usepackage{rotating}
\usepackage[normalem]{ulem}
\usepackage{amsmath}
\usepackage{amssymb}
\usepackage{capt-of}
\usepackage{hyperref}
\author{Jeason}
\date{\today}
\title{设计文档}
\hypersetup{
 pdfauthor={Jeason},
 pdftitle={设计文档},
 pdfkeywords={},
 pdfsubject={},
 pdfcreator={Emacs 29.1 (Org mode 9.7)}, 
 pdflang={English}}
\begin{document}

\maketitle


\section*{项目定位}
\label{sec:orga726524}
此项目是用来做长篇幅的文章书写的。对于长篇幅的书写,我不建议使用数据库来进行存储,这样是否会造成文字识别的篇幅过大,和不好保存呢?但是理论上来说数据库再怎么大也会比文本文件要小。项目目前的功能可以说是千疮百孔,完全不能拿来直接用。这个项目,是我在做的事情当中一个很认真在做的项目,也是很认真开展的一件事情。目前这个项目并不能满足我的预期,不管是哪方面,距离我想要的功能都比较遥远。

在设计项目的时候,是由于一些自身的感受才能延伸出来一些体会,而这些体会才会告诉你需要满足哪些特性。有了这些需要满足的特性,然后才有原型图和设计稿。我始终觉得要先有一些文字的说明,才能有这样一个原型图出来。否则你自己都不知道自己要做一个什么样的东西。

\textbf{软件比较要简洁,要清爽,文件的界面里面不需要有的东西都不需要,最好直接将界面的按钮操作都进行隐藏,然后使用快捷键进行操作。}

我不喜欢多余的东西,任何多余的东西都不需要,同样,我也不喜欢图片。我可以提供图片的生成,但是我不希望在我的文本文件里面显示出任何的图片出来。 \textbf{在首页,我喜欢以一个ASCII码的图标作为开始,然后我不需要图标,我不喜欢和不需要任何多余的东西} 。我不认为我现在需要做的是设计出一个具体的功能来,我认为现在是要给这个项目定一个基调,一个主旋律。以及我想通过这个项目表现出什么。

如果我是一个设计师,当我设计一把椅子的时候,我当然不会仅仅是为了坐,而是满足我自身的一些需要。 \uline{如果我设计了一把椅子,那这把椅子最重要的一定是有两点,一是满足自己的审美,符合我自己房间的风格主题;二则是满足我自己一些在生活当中想要的功能。} 我想要什么呢?我想要有一个放脚的地方,我想要有一个能让我缩在椅子里的放海绵垫的槽口。

同样的道理,当我设计这个项目的时候,我想要设计什么呢?我希望我做的这个东西能符合我目前对于软件一种主题的认识: \textbf{简洁、极至的简洁。} 而我自己定制化的功能,则就是我在写作的过程当中所遇到的一个又一个问题。在某种程度上,我喜欢我不仅仅提供的是一个软件,一个有着若干功能的软件,我喜欢能通过这个软件的这些功能,来教会使用者如何写作。

\begin{center}
\includegraphics[width=.9\linewidth]{/home/jeason/图片/2021-12-22 16-42-15屏幕截图.png}
\end{center}
\end{document}